\documentclass{anstrans}
%%%%%%%%%%%%%%%%%%%%%%%%%%%%%%%%%%%
\title{LEU+ to HALEU Transitions in Advanced Reactor Fuel Cycles}
\author{Nathan S. Ryan,$^{*}$ Kathryn D. Huff,$^{*}$ and Madicken Munk$^{*, \dagger}$}

\institute{
$^{*}$Advanced Reactors and Fuel Cycles Group, University of Illinois,
Urbana, IL, nsryan2@illinois.edu
\and
$^{\dagger}$Scientific Computing, Reactor Analysis and Modeling Group, Oregon State University, Corvallis, OR
}

% Optional disclaimer: remove this command to hide
% \disclaimer{Notice: this manuscript is a work of fiction. Any resemblance to
% actual articles, living or dead, is purely coincidental.}

%%%% packages and definitions (optional)
\usepackage{graphicx} % allows inclusion of graphics
\usepackage{booktabs} % nice rules (thick lines) for tables
\usepackage{microtype} % improves typography for PDF
\usepackage[acronym,toc]{glossaries}
\include{../acros}

\usepackage{algorithm}
\usepackage{algpseudocode}
\usepackage{array}

\usepackage{xspace}
\usepackage{multirow}

\newcommand{\cycamore}{\textsc{Cycamore}\xspace}
\newcommand{\cyclus}{\textsc{Cyclus}\xspace}
\newcommand{\SN}{S$_N$}
\renewcommand{\vec}[1]{\bm{#1}} %vector is bold italic
\newcommand{\vd}{\bm{\cdot}} % slightly bold vector dot
\newcommand{\grad}{\vec{\nabla}} % gradient
\newcommand{\ud}{\mathop{}\!\mathrm{d}} % upright derivative symbol

\begin{document}
%%%%%%%%%%%%%%%%%%%%%%%%%%%%%%%%%%%%%%%%%%%%%%%%%%%%%%%%%%%%%%%%%%%%%%%%%%%%%%%%
\section{Introduction}

In 2020, a \gls{haleu} workshop report led by Monica Regalbuto \cite{regalbuto_high_assay_2020} highlighted the unique regulatory challenges of establishing a \gls{haleu} fuel cycle in the \gls{usa}. It noted that part of enriching \gls{haleu} is first to produce \gls{leup}, defined as between 5\% and 10\% $^{235}U$ enrichment. The report notes that \gls{leup} facilities would fall under a similar category of regulations as our existing \gls{leu} fuel cycle, allowing existing enrichment servicers to leverage their experience and infrastructure before taking on the increased regulatory burden of producing \gls{haleu}. If a reactor could be redesigned to accommodate it, using \gls{leup} could delay the demand for \gls{haleu}.

% To enrich to up to \gls{haleu}, facilities such as TRISO-X LLC and Kairos Power Atlas Fuel Fabrication Facility must build . Thus, \gls{leup} is an attractive intermediary step for servicers wishing to minimize the size of a Category II facility (thereby reducing costs) as it is the same category we have historically licensed for \gls{leu} fuel enrichment.

% This report also notes that \gls{leup} could be used in existing \gls{leu} reactors, which would allow for a more rapid deployment of \gls{haleu} in the \gls{us}. This is an important point, as it would allow for a more rapid transition to \gls{haleu} without the need for new reactor designs or modifications to existing reactors.

% One of the primary advantages for a fuel cycle containing \gls{leup} is that the facility to produce it would fall under the same licensing category as \gls{leu} fuel. The \gls{nrc} defines a \textit{special nuclear material of low strategic significance} as meeting one of three criteria, the most notable of which for our purposes is "(3) 10,000 grams or more of uranium-235 (contained in uranium enriched above natural but less than 10 percent in the U–235 isotope)," \cite{nrc_catiii}. This facility definition is where the upper limit of the \gls{leup} range arises.

In this work, we use \cyclus to model the \gls{nfc} of a fuel cycle that
deploys Westinghouse AP1000s, \gls{xe}, and \gls{usnc} \gls{mmr}. After 10
years of operation with \gls{leup} fuel, the \gls{xe} reactors transition to
\gls{haleu} and new \glspl{mmr} are fueled with \gls{haleu}. We will evaluate
the impact of this transition on the delayed \gls{swu} and mass of \gls{haleu}
required to meet various demand growth scenarios.


%%%%%%%%%%%%%%%%%%%%%%%%%%%%%%%%%%%%%%%%%%%%%%%%%%%%%%%%%%%%%%%%%%%%%%%%%%%%%%%%
\section{\cyclus}

\cyclus \cite{huff_cyclus_intro_2016} is an agent-based \gls{nfc} simulator
that is versatile, open-source, and modular. The software achieves this
versatility through a series of generic archetypes, which represent facilities
or processes. Many standard fuel cycle facilities have
been implemented in the \cycamore repository \cite{Carlsen_cycamore_2014},
which holds technology-agnostic archetypes for material sources, material
sinks, enrichment services, separations capabilities, storage services, and a
generic reactor. Outside of \cycamore, the community has created a catalog of
archetypes that can simulate a wide range of fuel cycle activities (e.g.,
material diversion PYRE \cite{westphal_modeling_2019} or fuel burnup with
OpenMCyclus \cite{openmcyclus_paper}).

% \begin{figure}[!htbp]
%   \centering
%   \includegraphics[width=0.4\textwidth]{../images/cyclus_logo.png}
%   \label{fig:cyclus_logo}
% \end{figure}

\section{Scenarios}
We will deploy reactors to meet energy demands drawn from \gls{eia} \cite{eia_aeo_2023} and \gls{doe} projections \cite{julie_liftoff_pathways_2024}, shown in Table \ref{tab:demand_scenarios}. The low growth scenarios correspond to 5\%, 10\%, and 15\% increases in energy demand by 2050, and the high growth scenarios correspond to a doubling and a tripling of nuclear energy by 2050.

\begin{table}[h]
  \centering
  \caption{Demand Growth Scenarios}
  \label{tab:demand_scenarios}
  \begin{tabular}{l c c}
      \hline
      \textbf{Demand Growth} & \textbf{Year-to-Year Increase} & \textbf{Source}\\
      \hline
      No Growth & 0\% & N/A\\
      Low Growth & 0.17\% & \cite{eia_aeo_2023}\\
       & 0.5\% & \cite{eia_aeo_2023}\\
       & 1\% & \cite{eia_aeo_2023}\\
      High Growth & 3.5\% & \cite{julie_liftoff_pathways_2024} \\
       & 5.6\% & \cite{julie_liftoff_pathways_2024}\\
      \hline
  \end{tabular}
\end{table}

In Table \ref{tab:deployment_schemes}, we describe the three schemes we used to deploy reactors to meet the energy demand. Through the greedy scheme, we are not attempting to capture the complexity of the problem but rather to explore the implications of deploying a minimal number of reactors to meet the demand. Contrasty, the random scheme is a proxy for the complexity of the real-world deployment problem; however, it does not include the nuance of how individual deployments meet an end user's needs, which will drive the strategic decisions that utilities and ratepayers behind the meter make in their reactor choices. Combining the random and greedy schemes allows us to inject some uncertainty around which reactor will be deployed at any given time while ensuring meeting the demand efficiently.

\begin{table}[h]
  \centering
  \caption{Deployment Schemes}
  \label{tab:deployment_schemes}
  \begin{tabular}{>{\raggedright}p{0.24\linewidth}>{\raggedright\arraybackslash}p{0.60\linewidth}}
      \hline
      Scheme & Description \\
      \hline
      Greedy Deployment & Deploy the largest
      reactor first at each time step, fill in the remaining capacity with
      the next smallest, and so on. \vspace{2mm}\\
      Random Deployment & Use the date and hour as a seed to sample the
      reactor list randomly. \vspace{2mm}\\
      Initially Random, Greedy Deployment & Randomly deploy reactors until
      a reactor bigger than the remaining capacity is proposed for each time step,
      then fill the remaining capacity with the greedy algorithm. \\
      \hline
  \end{tabular}
\end{table}

% \begin{algorithm}
% \caption{Greedy Reactor Deployment Algorithm}
% \begin{algorithmic}[1]
%     \State Initialize demand
%     \While{demand exists}
%         \State Select the largest reactor that does not exceed demand
%         \State Deploy reactors until the next reactor exceeds demand
%         \State Update demand
%     \EndWhile
% \end{algorithmic}
% \end{algorithm}

% \begin{algorithm}
%   \caption{Random Reactor Deployment Algorithm}
%   \begin{algorithmic}[1]
%       \State Initialize demand
%       \While{demand exists}
%           \State Randomly deploy a reactor that does not exceed demand
%           \State Update demand
%       \EndWhile
%   \end{algorithmic}
%   \end{algorithm}

%   \begin{algorithm}
%     \caption{Random + Greedy Reactor Deployment Algorithm}
%     \begin{algorithmic}[1]
%         \State Initialize demand
%         \While{demand exists}
%             \State Randomly deploy a reactor
%             \If{demand is exceeded}
%                 \State Remove last reactor
%                 \If{demand still exists}
%                     \State Select the largest reactor that does not exceed demand
%                     \State Deploy until the next reactor exceeds demand
%                     \State Update demand
%                 \EndIf
%             \EndIf
%         \EndWhile
%     \end{algorithmic}
%     \end{algorithm}

% \begin{subequations} \label{eqs:fullTransport}
% \begin{multline} \label{eq:fullTransportVol}
%   \vec{\Omega}\vd \grad \psi(\vec{x}, \vec{\Omega})
%   + \sigma(\vec{x}) \psi (\vec{x}, \vec{\Omega})
% \\ =
%   \frac{\sigma_s(\vec{x})}{4\pi} \int_{4\pi} \psi(\vec{x},\vec{\Omega}')
%   \ud\Omega' + \frac{q(\vec{x})}{4\pi}
%   \equiv \frac{1}{4\pi} Q(\vec{x}) \,,
% \end{multline}
% inside $\vec{x} \in V$, $\vec{\Omega} \in 4\pi$, with an incident boundary
% condition
% \begin{equation} \label{eq:fullTransportBndy}
%   \psi(\vec{x}, \vec{\Omega}) = \psi^b(\vec{x}, \vec{\Omega}) \,,
%  \quad \vec{x} \in \partial V, \ \vec{\Omega} \vd \vec{n} < 0\,.
% \end{equation}
% \end{subequations}

%%%%%%%%%%%%%%%%%%%%%%%%%%%%%%%%%%%%%%%%%%%%%%%%%%%%%%%%%%%%%%%%%%%%%%%%%%%%%%%%
% \section{Results and Analysis}
% Table \ref{tab:enrichment_levels} shows the various levels of enrichment for uranium that we will use in this work.

% \begin{table}[!htbp]
%    \centering
%    \caption{Enrichment levels and their ranges.}
%    \label{tab:enrichment_levels}
%    \begin{tabular}{c c}
%       \hline
%       \textbf{Enrichment Level} & \textbf{Range [\%  $^{235}$U]} \\
%       \hline
%       Natural & < 0.711 \\
%       \gls{leu} & 0.711-5 \\
%       \gls{leup} & 5-10 \\
%       \gls{haleu} & 10-20 \\
%       % \gls{heu} & $\geq$ 20  \\
%       \hline
%    \end{tabular}
% \end{table}

%%%%%%%%%%%%%%%%%%%%%%%%%%%%%%%%%%%%%%%%%%%%%%%%%%%%%%%%%%%%%%%%%%%%%%%%%%%%%%%%
% \subsection{Subsection Goes Here (Heading B)}
% The user must manually capitalize initial letters of a subsection heading.

% For those who like equations in their papers, \LaTeX\ is a good choice. Here is
% an equation for the Marshak diffusion boundary condition:
% \begin{equation} \label{eq:marshak}
%   4 J^- = \phi + 2 D \vec{n} \vd \grad \phi \,.
% \end{equation}
% If we so choose, we can effortlessly reference the equation later.

% Another paragraph starts with Eq.~\eqref{eq:marshak} and sets $J^-$ to zero, a
% vacuum boundary condition:
% \begin{equation*}
%   0 = \phi + \frac{2}{3} \frac{1}{\sigma} \vec{n} \vd \grad \phi \,.
% \end{equation*}
% The extrapolation distance is $2/3$. A more detailed asymptotic analysis yields
% an extrapolation distance of about $0.71045$.



% Later on, we can include a table, even one that spans two columns such as
% Table~\ref{tab:widetable}.
%%%%%%%%%%%%%%%%%%%%%%%%%%%%%%%%%%%%%%%%
% \begin{table*}[htb]
%   \centering
%   \caption{Example of a Really Wide Table that Might Not Normally Fit in the Document}
%   \begin{tabular}{llllllllll}\toprule
%       & $\phi_T(0)$      & $\phi_T(10)$      & $\phi_T(20)$      &
%       $\phi_D(0)$      & $\phi_D(10)$      & $\phi_D(20)$      & $\rho$      &
%       $\varepsilon$      & $N_\text{it}$
% \\ \midrule
% $c=0.999$  & 0.9038 & 20.63 & 31.24 & 0.9087 & 20.63 & 31.23 & 0.2192 & $10^{-7}$ & 15
% \\
% $c=0.990$  & 0.3675 & 13.04 & 24.7 & 0.3696 & 13.04 & 24.69 & 0.2184 & $10^{-7}$ & 15
% \\
% $c=0.900$  & 0.009909 & 4.776 & 17.64 & 0.009984 & 4.786 & 17.63 & 0.2118 & $10^{-7}$ & 14
% \\
% $c=0.500$  & $6.069\times 10^{-5}$ & 2.212 & 15.53 & 6.213$\times 10^{-5}$ & 2.239 & 15.53 & 0.2068 & $10^{-7}$ & 13
% \\
% \bottomrule
% \end{tabular}
%   \label{tab:widetable}
% \end{table*}
%%%%%%%%%%%%%%%%%%%%%%%%%%%%%%%%%%%%%%%%
% Notice how the table reference uses a Roman numeral
% for its numbering scheme, whereas the figure reference uses an Arabic numeral.
% For one-column tables, use the \verb|table| environment; two-column tables use
% \verb|table*|. The same applies to figures.

%%%%%%%%%%%%%%%%%%%%%%%%%%%%%%%%%%%%%%%%%%%%%%%%%%%%%%%%%%%%%%%%%%%%%%%%%%%%%%%%
% \section{Conclusions (Heading A)}

% The included ANS style file and this clear example file are a panacea for
% the hours of headache that invariably results from formatting a document in
% Microsoft Word.

%%%%%%%%%%%%%%%%%%%%%%%%%%%%%%%%%%%%%%%%%%%%%%%%%%%%%%%%%%%%%%%%%%%%%%%%%%%%%%%%
\appendix
% \section{Appendix}

% Numbering in the appendix is different:
% \begin{equation} \label{eq:appendix}
%   2 + 2 = 5\,.
% \end{equation}
% and another equation:
% \begin{equation} \label{eq:appendix2}
%   a + b = c\,.
% \end{equation}

%%%%%%%%%%%%%%%%%%%%%%%%%%%%%%%%%%%%%%%%%%%%%%%%%%%%%%%%%%%%%%%%%%%%%%%%%%%%%%%%
\section{Acknowledgments}
This research was performed, in part, using funding received from the DOE
Office of Nuclear Energy's Nuclear Energy University Program (Project 23-29656
DE-NE0009390) 'Illuminating Emerging Supply Chain and Waste Management
Challenges'.

%%%%%%%%%%%%%%%%%%%%%%%%%%%%%%%%%%%%%%%%%%%%%%%%%%%%%%%%%%%%%%%%%%%%%%%%%%%%%%%%
\bibliographystyle{ans}
\bibliography{../bibliography}
\end{document}

