\documentclass{anstrans}
%%%%%%%%%%%%%%%%%%%%%%%%%%%%%%%%%%%
\title{the title}
\author{Nathan S. Ryan,$^{*}$ Kathryn D. Huff,$^{*}$ and Madicken Munk,$^{\dagger}$}

\institute{
$^{*}$Advanced Reactors and Fuel Cycles Group, University of Illinois,
Urbana, IL, nsryan2@illinois.edu
\and
$^{\dagger}$Scientific Computing, Reactor Analysis and Modeling Group, Oregon State University, Corvallis, OR
}

% Optional disclaimer: remove this command to hide
% \disclaimer{Notice: this manuscript is a work of fiction. Any resemblance to
% actual articles, living or dead, is purely coincidental.}

%%%% packages and definitions (optional)
\usepackage{graphicx} % allows inclusion of graphics
\usepackage{booktabs} % nice rules (thick lines) for tables
\usepackage{microtype} % improves typography for PDF
\usepackage[acronym,toc]{glossaries}
\include{../acros}

\usepackage{algorithm}
\usepackage{algpseudocode}

\newcommand{\SN}{S$_N$}
\renewcommand{\vec}[1]{\bm{#1}} %vector is bold italic
\newcommand{\vd}{\bm{\cdot}} % slightly bold vector dot
\newcommand{\grad}{\vec{\nabla}} % gradient
\newcommand{\ud}{\mathop{}\!\mathrm{d}} % upright derivative symbol

\begin{document}
%%%%%%%%%%%%%%%%%%%%%%%%%%%%%%%%%%%%%%%%%%%%%%%%%%%%%%%%%%%%%%%%%%%%%%%%%%%%%%%%
\section{Introduction}

As we began developing advanced reactor programs and demonstrations, increasingly fueled by \gls{haleu}, the issues of establishing an appropriate fuel cycle have come to the forefront. \gls{leu} has been the standard for light water reactors, but as we look to advanced reactors, we must consider the implications of using \gls{haleu} and how to best implement it. One of the most pressing questions is whether we should enrich to \gls{leup} or \gls{haleu}. This work will explore the implications of each option.

One of the primary advantages for a fuel cycle containing \gls{leup} is that the facility to produce it would fall under the same licensing category as \gls{leu} fuel. The \gls{nrc} defines a \textit{special nuclear material of low strategic significance} as meeting one of three criteria, the most notable of which for our purposes is "(3) 10,000 grams or more of uranium-235 (contained in uranium enriched above natural but less than 10 percent in the U–235 isotope)," \cite{nrc_catiii}. This facility definition is where the upper limit of the \gls{leup} range arises.

To enrich to up to \gls{haleu}, facilities such as TRISO-X LLC and Kairos Power Atlas Fuel Fabrication Facility must move up a category to \textit{special nuclear material of moderate strategic significance} (Category II). Thus, \gls{leup} is an attractive intermediary step for servicers wishing to minimize the size of a Category II facility (thereby reducing costs) as it is the same category we have historically licensed for \gls{leu} fuel enrichment.

%%%%%%%%%%%%%%%%%%%%%%%%%%%%%%%%%%%%%%%%%%%%%%%%%%%%%%%%%%%%%%%%%%%%%%%%%%%%%%%%
\section{Theory}

\begin{figure}[!htbp]
  \centering
  \includegraphics[width=0.4\textwidth]{../images/cyclus_logo.png}
  \label{fig:cyclus_logo}
\end{figure}

\begin{table}[H]
  \centering
  \caption{Deployment Schemes}
  \label{tab:deployment_schemes}
  \begin{tabular}{p{0.24\linewidth} p{0.60\linewidth}}
      \hline
      Scheme & Description \\
      \hline
      Greedy Deployment & Deploy the largest
      reactor first at each time step, fill in the remaining capacity with
      the next smallest, and so on. \\
      Random Deployment & Uses a date and hour as seed to sample the
      reactors list randomly. \\
      Initially Random, Greedy Deployment & Randomly deploys reactors until
      a reactor bigger than the remaining capacity is proposed for each time step,
      then fills the remaining capacity with the greedy algorithm. \\
      \hline
  \end{tabular}
\end{table}

% \begin{algorithm}
% \caption{Greedy Reactor Deployment Algorithm}
% \begin{algorithmic}[1]
%     \State Initialize demand
%     \While{demand exists}
%         \State Select the largest reactor that does not exceed demand
%         \State Deploy reactors until the next reactor exceeds demand
%         \State Update demand
%     \EndWhile
% \end{algorithmic}
% \end{algorithm}

% \begin{algorithm}
%   \caption{Random Reactor Deployment Algorithm}
%   \begin{algorithmic}[1]
%       \State Initialize demand
%       \While{demand exists}
%           \State Randomly deploy a reactor that does not exceed demand
%           \State Update demand
%       \EndWhile
%   \end{algorithmic}
%   \end{algorithm}

%   \begin{algorithm}
%     \caption{Random + Greedy Reactor Deployment Algorithm}
%     \begin{algorithmic}[1]
%         \State Initialize demand
%         \While{demand exists}
%             \State Randomly deploy a reactor
%             \If{demand is exceeded}
%                 \State Remove last reactor
%                 \If{demand still exists}
%                     \State Select the largest reactor that does not exceed demand
%                     \State Deploy until the next reactor exceeds demand
%                     \State Update demand
%                 \EndIf
%             \EndIf
%         \EndWhile
%     \end{algorithmic}
%     \end{algorithm}

% \begin{subequations} \label{eqs:fullTransport}
% \begin{multline} \label{eq:fullTransportVol}
%   \vec{\Omega}\vd \grad \psi(\vec{x}, \vec{\Omega})
%   + \sigma(\vec{x}) \psi (\vec{x}, \vec{\Omega})
% \\ =
%   \frac{\sigma_s(\vec{x})}{4\pi} \int_{4\pi} \psi(\vec{x},\vec{\Omega}')
%   \ud\Omega' + \frac{q(\vec{x})}{4\pi}
%   \equiv \frac{1}{4\pi} Q(\vec{x}) \,,
% \end{multline}
% inside $\vec{x} \in V$, $\vec{\Omega} \in 4\pi$, with an incident boundary
% condition
% \begin{equation} \label{eq:fullTransportBndy}
%   \psi(\vec{x}, \vec{\Omega}) = \psi^b(\vec{x}, \vec{\Omega}) \,,
%  \quad \vec{x} \in \partial V, \ \vec{\Omega} \vd \vec{n} < 0\,.
% \end{equation}
% \end{subequations}

%%%%%%%%%%%%%%%%%%%%%%%%%%%%%%%%%%%%%%%%%%%%%%%%%%%%%%%%%%%%%%%%%%%%%%%%%%%%%%%%
\section{Results and Analysis}
Table \ref{tab:enrichment_levels} shows the various levels of enrichment for uranium that we will use in this work.

\begin{table}[!htbp]
   \centering
   \caption{Enrichment levels and their ranges.}
   \label{tab:enrichment_levels}
   \begin{tabular}{c c}
      \hline
      \textbf{Enrichment Level} & \textbf{Range [\%  $^{235}$U]} \\
      \hline
      Natural & < 0.711 \\
      \gls{leu} & 0.711-5 \\
      \gls{leup} & 5-10 \\
      \gls{haleu} & 10-20 \\
      % \gls{heu} & $\geq$ 20  \\
      \hline
   \end{tabular}
\end{table}

%%%%%%%%%%%%%%%%%%%%%%%%%%%%%%%%%%%%%%%%%%%%%%%%%%%%%%%%%%%%%%%%%%%%%%%%%%%%%%%%
\subsection{Subsection Goes Here (Heading B)}
The user must manually capitalize initial letters of a subsection heading.

For those who like equations in their papers, \LaTeX\ is a good choice. Here is
an equation for the Marshak diffusion boundary condition:
\begin{equation} \label{eq:marshak}
  4 J^- = \phi + 2 D \vec{n} \vd \grad \phi \,.
\end{equation}
If we so choose, we can effortlessly reference the equation later.

Another paragraph starts with Eq.~\eqref{eq:marshak} and sets $J^-$ to zero, a
vacuum boundary condition:
\begin{equation*}
  0 = \phi + \frac{2}{3} \frac{1}{\sigma} \vec{n} \vd \grad \phi \,.
\end{equation*}
The extrapolation distance is $2/3$. A more detailed asymptotic analysis yields
an extrapolation distance of about $0.71045$.



% Later on, we can include a table, even one that spans two columns such as
% Table~\ref{tab:widetable}.
%%%%%%%%%%%%%%%%%%%%%%%%%%%%%%%%%%%%%%%%
% \begin{table*}[htb]
%   \centering
%   \caption{Example of a Really Wide Table that Might Not Normally Fit in the Document}
%   \begin{tabular}{llllllllll}\toprule
%       & $\phi_T(0)$      & $\phi_T(10)$      & $\phi_T(20)$      &
%       $\phi_D(0)$      & $\phi_D(10)$      & $\phi_D(20)$      & $\rho$      &
%       $\varepsilon$      & $N_\text{it}$
% \\ \midrule
% $c=0.999$  & 0.9038 & 20.63 & 31.24 & 0.9087 & 20.63 & 31.23 & 0.2192 & $10^{-7}$ & 15
% \\
% $c=0.990$  & 0.3675 & 13.04 & 24.7 & 0.3696 & 13.04 & 24.69 & 0.2184 & $10^{-7}$ & 15
% \\
% $c=0.900$  & 0.009909 & 4.776 & 17.64 & 0.009984 & 4.786 & 17.63 & 0.2118 & $10^{-7}$ & 14
% \\
% $c=0.500$  & $6.069\times 10^{-5}$ & 2.212 & 15.53 & 6.213$\times 10^{-5}$ & 2.239 & 15.53 & 0.2068 & $10^{-7}$ & 13
% \\
% \bottomrule
% \end{tabular}
%   \label{tab:widetable}
% \end{table*}
%%%%%%%%%%%%%%%%%%%%%%%%%%%%%%%%%%%%%%%%
% Notice how the table reference uses a Roman numeral
% for its numbering scheme, whereas the figure reference uses an Arabic numeral.
% For one-column tables, use the \verb|table| environment; two-column tables use
% \verb|table*|. The same applies to figures.

%%%%%%%%%%%%%%%%%%%%%%%%%%%%%%%%%%%%%%%%%%%%%%%%%%%%%%%%%%%%%%%%%%%%%%%%%%%%%%%%
\section{Conclusions (Heading A)}

The included ANS style file and this clear example file are a panacea for
the hours of headache that invariably results from formatting a document in
Microsoft Word.

%%%%%%%%%%%%%%%%%%%%%%%%%%%%%%%%%%%%%%%%%%%%%%%%%%%%%%%%%%%%%%%%%%%%%%%%%%%%%%%%
\appendix
\section{Appendix}

Numbering in the appendix is different:
\begin{equation} \label{eq:appendix}
  2 + 2 = 5\,.
\end{equation}
and another equation:
\begin{equation} \label{eq:appendix2}
  a + b = c\,.
\end{equation}

%%%%%%%%%%%%%%%%%%%%%%%%%%%%%%%%%%%%%%%%%%%%%%%%%%%%%%%%%%%%%%%%%%%%%%%%%%%%%%%%
\section{Acknowledgments}
This material is based upon work supported by a Department of Energy Nuclear
Energy University Programs Graduate Fellowship.

%%%%%%%%%%%%%%%%%%%%%%%%%%%%%%%%%%%%%%%%%%%%%%%%%%%%%%%%%%%%%%%%%%%%%%%%%%%%%%%%
\bibliographystyle{ans}
\bibliography{../bibliography}
\end{document}

